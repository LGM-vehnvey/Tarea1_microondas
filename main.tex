%%%%%%%%%%%%%%%%%%%%%%%%%%%%% Define Article %%%%%%%%%%%%%%%%%%%%%%%%%%%%%%%%%%
\documentclass{article}   % two side printing
%%%%%%%%%%%%%%%%%%%%%%%%%%%%%%%%%%%%%%%%%%%%%%%%%%%%%%%%%%%%%%%%%%%%%%%%%%%%%%%

%%%%%%%%%%%%%%%%%%%%%%%%%%%%% Using Packages %%%%%%%%%%%%%%%%%%%%%%%%%%%%%%%%%%
\usepackage{geometry}
\usepackage{graphicx}
\usepackage{amssymb}
\usepackage{amsmath}
\usepackage{amsthm}
\usepackage{empheq}
\usepackage{mdframed}
\usepackage{booktabs}
\usepackage{lipsum}
\usepackage{color}
\usepackage{psfrag}
\usepackage{pgfplots}   % For plotting beautiful graphs
\usepackage{bm}
\usepackage[spanish]{babel}
\usepackage{biblatex} 
\usepackage{csquotes} 
\usepackage{setspace}
\usepackage{multicol}  
\usepackage[skip=3pt plus 1pt, indent=30pt]{parskip}    % Setting space between paragraphs and indent
\usepackage[T1]{fontenc}    % Output font encoding for international characters
\usepackage{helvet}        % Selecting font family
\usepackage{ragged2e}      % For text alignment
\usepackage{adjustbox}       % For defining new environments
\usepackage{fancyhdr}       % For defining headers and footers
\usepackage[cochineal]{newtxmath}
\usepackage{caption}
\usetikzlibrary{intersections}
\usetikzlibrary{decorations.text}
\usetikzlibrary{decorations.pathreplacing}
%%%%%%%%%%%%%%%%%%%%%%%%%%%%%%%%%%%%%%%%%%%%%%%%%%%%%%%%%%%%%%%%%%%%%%%%%%%%%%%

% Other Settings
\newcommand*{\freq}{\mathord{\mathit{f}}}
%\let\cleardoublepage=\clearpage
%\renewcommand{\baselinestretch}{1.5}
%%%%%%%%%%%%%%%%%%%%%%%%%% Page Setting %%%%%%%%%%%%%%%%%%%%%%%%%%%%%%%%%%%%%%%
\geometry{a4paper, textwidth=19cm, textheight=28.5cm, top=0.1cm, headheight=0.1cm}  % Setting page size
\graphicspath{{images/}}    % Setting path for images
\addbibresource{bibliography.bib}   % Setting path for bibliography
%%%%%%%%%%%%%%%%%%%%%%%%%% Define some useful colors %%%%%%%%%%%%%%%%%%%%%%%%%%
\definecolor{ocre}{RGB}{243,102,25}
\definecolor{mygray}{RGB}{243,243,244}
\definecolor{deepGreen}{RGB}{26,111,0}
\definecolor{shallowGreen}{RGB}{235,255,255}
\definecolor{deepBlue}{RGB}{61,124,222}
\definecolor{shallowBlue}{RGB}{235,249,255}
%%%%%%%%%%%%%%%%%%%%%%%%%%%%%%%%%%%%%%%%%%%%%%%%%%%%%%%%%%%%%%%%%%%%%%%%%%%%%%%

%%%%%%%%%%%%%%%%%%%%%%%%%% Define an orange box command %%%%%%%%%%%%%%%%%%%%%%%%
\newcommand\orangebox[1]{\fcolorbox{ocre}{mygray}{\hspace{1em}#1\hspace{1em}}}
%%%%%%%%%%%%%%%%%%%%%%%%%%%%%%%%%%%%%%%%%%%%%%%%%%%%%%%%%%%%%%%%%%%%%%%%%%%%%%%

%%%%%%%%%%%%%%%%%%%%%%%%%%%% English Environments %%%%%%%%%%%%%%%%%%%%%%%%%%%%%
\newtheoremstyle{mytheoremstyle}{3pt}{3pt}{\normalfont}{0cm}{\rmfamily\bfseries}{}{1em}{{\color{black}\thmname{#1}~\thmnumber{#2}}\thmnote{\,--\,#3}}
\newtheoremstyle{myproblemstyle}{3pt}{3pt}{\normalfont}{0cm}{\rmfamily\bfseries}{}{1em}{{\color{black}\thmname{#1}~\thmnumber{#2}}\thmnote{\,--\,#3}}
\theoremstyle{mytheoremstyle}
\newmdtheoremenv[linewidth=1pt,backgroundcolor=shallowGreen,linecolor=deepGreen,leftmargin=0pt,innerleftmargin=20pt,innerrightmargin=20pt,]{theorem}{Theorem}[section]
\theoremstyle{mytheoremstyle}
\newmdtheoremenv[linewidth=1pt,backgroundcolor=shallowBlue,linecolor=deepBlue,leftmargin=0pt,innerleftmargin=20pt,innerrightmargin=20pt,]{definition}{Definition}[section]
\theoremstyle{myproblemstyle}
\newmdtheoremenv[linecolor=black,leftmargin=0pt,innerleftmargin=10pt,innerrightmargin=10pt,]{problem}{Problem}[section]
%%%%%%%%%%%%%%%%%%%%%%%%%%%%%%%%%%%%%%%%%%%%%%%%%%%%%%%%%%%%%%%%%%%%%%%%%%%%%%%

%%%%%%%%%%%%%%%%%%%%%%%%%%%%%%% Plotting Settings %%%%%%%%%%%%%%%%%%%%%%%%%%%%%
\usepgfplotslibrary{colorbrewer}
\pgfplotsset{width=8cm,compat=1.9}
%%%%%%%%%%%%%%%%%%%%%%%%%%%%%%%%%%%%%%%%%%%%%%%%%%%%%%%%%%%%%%%%%%%%%%%%%%%%%%%

%%%%%%%%%%%%%%%%%%%%%%%%%%%%%%% Title & Author %%%%%%%%%%%%%%%%%%%%%%%%%%%%%%%%
\title{Caracterización de la atenuación y la permitividad en líneas de transmisión}
\author{Luis Guillermo Macias Rojas}
%%%%%%%%%%%%%%%%%%%%%%%%%%%%%%%%%%%%%%%%%%%%%%%%%%%%%%%%%%%%%%%%%%%%%%%%%%%%%%%

%%%%%%%%%%%%%%%%%%%%%%%%%%%%%%% Header & Footer %%%%%%%%%%%%%%%%%%%%%%%%%%%%%%%
\pagestyle{fancy}  % Setting page style
\fancyhf{}
\fancyhead[L]{\text{Luis Guillermo Macias Rojas}}  % Setting header
%%%%%%%%%%%%%%%%%%%%%%%%%%%%%%%%%%%%%%%%%%%%%%%%%%%%%%%%%%%%%%%%%%%%%%%%%%%%%%%

\begin{document}
    \maketitle

    \fontfamily{phv}\selectfont % Selecting font family
    \noindent
    \textbf{Resumen:} En este trabajo se presenta la metodología y resultados obtenidos en la caracterización de la atenuación 
    ($\alpha$) y la permitividad relativa ($\epsilon_{\text{r'}}$) en líneas de transmisión a partir de datos experimentales
    obtenidos en el laboratorio. Para la extracción de las componentes de disipación ($\alpha_{\text{c}}$, $\alpha_{\text{d}}$) 
    y dispersión ($\beta{\text{c}}$, $\beta{\text{d}}$) se realizó una regresión lineal considerando un modelo donde el conductor se 
    supone liso y la línea de transmisión de bajas pérdidas. Los resultados obtenidos muestran una frecuencia de \textit{cross-over} 
    $\freq_{\theta}$ = 27.42 GHz y una permitividad relativa $\epsilon_{\text{r'}}$ que se satura alrededor de 1.87.
    \noindent\begin{minipage}{0.49\textwidth}   %uses 60% of the page
        {\centering\section*{\large Introducción}}

        La disipación ($\alpha$) y la dispersión ($\beta$) en líneas de transmisión son fenómenos que se presentan
        en la propagación de ondas electromagnéticas a través de un medio conductor. Este trabajo describe la extracción de estos 
        parámetros fundamentales en la caracterización de líneas de transmisión.
        
        {\centering\section*{\large Metodología}}

        La extracción de $\alpha_{\text{c}}$, $\alpha_{\text{d}}$, $\beta_{\text{c}}$ y $\beta_{\text{d}}$ parte de un modelo para 
        líneas de transmisión de bajas pérdidas que puede ser descrito por las ecuaciones \eqref{eq:alpha_model} y \eqref{eq:beta_model}.

        \begin{equation}
            \alpha = \underbrace{\frac{K_s \sqrt{\freq}}{2 Z(s)_{\mathbb{R}}}}_{\alpha_c} + \underbrace{\frac{G}{2} Z(s)_{\mathbb{R}}}_{\alpha_d}
            \label{eq:alpha_model}
        \end{equation}

        \begin{equation}
            \beta = \underbrace{\frac{K_s \sqrt{\freq}}{2 Z(s)_{\mathbb{R}}}}_{\beta_c} + \underbrace{\frac{\omega}{\nu_p}}_{\beta_d}
            \qquad \beta_d = \frac{\omega \sqrt{\epsilon_{r'}}}{c}
            \label{eq:beta_model}
        \end{equation}

        Nótese que $\alpha_{\text{c}}$ y $\beta_{\text{c}}$ son iguales debido a que el conductor se considera liso.

        Debido a que todos los parámetros excepto $\freq$ son constantes pueden ser agrupados en una sola constante
        (A$_{\alpha}$, B$_{\alpha}$, A$_{\beta}$ y B$_{\beta}$) para simplificar términos, si además se reordenan \eqref{eq:alpha_model} 
        y \eqref{eq:beta_model} conforme a la ecuación de la recta (y = mx + b) pueden reescribirse como \eqref{eq:alpha_coeff} y 
        \eqref{eq:beta_coeff} respectivamente. 

        \begin{equation}
            \underbrace{\frac{\alpha}{\sqrt{\freq}}}_{y} = \underbrace{A_{\alpha}}_{b} + \underbrace{B_{\alpha} \sqrt{\freq}}_{mx}
            \label{eq:alpha_coeff}
        \end{equation}

        \begin{equation}
            \underbrace{\frac{\beta}{\sqrt{\freq}}}_{y} = \underbrace{A_{\beta}}_{b} + \underbrace{B_{\beta} \sqrt{\freq}}_{mx}
            \label{eq:beta_coeff}
        \end{equation}

        Posteriormente se pueden hallar los coeficientes de \eqref{eq:alpha_coeff} y \eqref{eq:beta_coeff} mediante una regresión
        lineal y de esa manera obtener las componentes del dieléctrico y del conductor.

        Conociendo los coeficientes de dispersión se puede hallar la permitividad relativa ($\epsilon_{r'}$) despejándola de la ecuación 
        \eqref{eq:beta_model} tal como se muestra en \eqref{eq:permittivity}.

        \begin{equation}
            \epsilon_{r'} = \left[ c \left(\frac{\beta - A_{\beta} \sqrt{\freq}}{\omega} \right) \right]^2
            \label{eq:permittivity}
        \end{equation}

    \end{minipage}
    \hspace{0.38 cm}
    \begin{minipage}{0.49\textwidth}   %uses 40% of the page
        \begin{adjustbox}{width=\textwidth}
            \begin{tikzpicture}
                \begin{axis}[
                    legend entries={$\alpha_{\text{c}}$, $\alpha_{\text{d}}$},
                    legend pos=north west,
                    xlabel={Frecuencia (GHz)},
                    ylabel={Atenuación ($\frac{\text{Np}}{\text{m}}$)},
                    %grid=major,
                    %grid style={dashed,gray!30},
                    %xtick={0, 5, 10, 15, 20, 25, 30, 35, 40},
                    %ytick={0, 0.25, 0.5, 0.75, 1, 1.25, 1.5, 1.75, 2},
                    axis lines=left,]
                    \addplot+[
                        deepBlue,
                        mark size=0.2pt,  % Setting mark size
                    ]
                    table[x index=0, y index=1]
                    {data/processed_data.txt}   % Reading data from dataFile.txt
                    node[left, pos=0.9, black] {$f_\theta$};   % Adding a node
                    \addplot+[
                        red,
                        mark size=0.2pt,  % Setting mark size
                    ]
                    table[x index=0, y index=2]
                    {data/processed_data.txt};
                    \draw[dashed, black] (274.2, 0) -- (274.2, 160);
                \end{axis}
            \end{tikzpicture}
        \end{adjustbox}
        \captionof{figure}{Disipación en función de la frecuencia.}    % Caption of the figure outside of float enviroment
        \label{fig:atenuacion}

        {\centering\section*{\large Resultados}}

        La figura \ref{fig:atenuacion} muestra las componentes de disipación ($\alpha_{\text{c}}$, $\alpha_{\text{d}}$) en función de la
        frecuencia. Se observa que la frecuencia de \textit{cross-over} $\freq_{\theta}$ es de 27.42 GHz. La figura \ref{fig:permitividad} muestra la permitividad relativa ($\epsilon_{\text{r'}}$) en función de la frecuencia.
        Se observa que a partir de 6 GHz comienza a saturarse en un valor cercano a 1.87 (semejante al valor constante que se habría obtenido
        erróneamente si no se hubiera considerado la componente de dispersión del conductor $\beta_c$).

        \begin{adjustbox}{width=\textwidth}
            \begin{tikzpicture}
                \begin{axis}[
                    legend entries={$\epsilon_{\text{r'}}$ correcta, $\epsilon_{\text{r'}}$ errónea},
                    legend pos=north west,
                    xlabel={Frecuencia (GHz)},
                    ylabel={$\epsilon_{\text{r'}}$},
                    %grid=major,
                    %grid style={dashed,gray!30},
                    %xtick={0, 5, 10, 15, 20, 25, 30, 35, 40},
                    %ytick={0, 0.25, 0.5, 0.75, 1, 1.25, 1.5, 1.75, 2},
                    axis lines=left,]
                    \addplot+[
                        deepBlue,
                        mark size=0.2pt,  % Setting mark size
                    ]
                    table[x index=0, y index=3]
                    {data/processed_data.txt};   % Reading data from dataFile.txt
                    %\addlegendentry{$\alpha_{\text{c}}$}
                    \addplot+[
                        orange,
                        mark size=0.2pt,  % Setting mark size
                    ]
                    table[x index=0, y index=4]
                    {data/processed_data.txt};
                \end{axis}
            \end{tikzpicture}
        \end{adjustbox}
        \captionof{figure}{Permitividad relativa en función de la frecuencia.}    % Caption of the figure outside of float enviroment
        \label{fig:permitividad}

    \end{minipage}
    %\printbibliography  % Print bibliography
\end{document}